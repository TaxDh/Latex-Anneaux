\documentclass[12pt,a4paper]{article}

\usepackage[left=1.5cm,right=1.5cm,top=1.5cm,bottom=1.5cm]{geometry}
\usepackage[T1]{fontenc}
\usepackage[utf8]{inputenc}
\usepackage[french]{babel}
\usepackage{amssymb}
\newtheorem{theorem}{Theorem}
\newtheorem{definition}{Definition}[section]
\usepackage{setspace}
\onehalfspacing
\usepackage{amsthm}
\usepackage{amsmath}
\begin{document}

%nouvelle commande

\newcommand{\znz}{\mathbb{Z} / n \mathbb{Z} }
\newcommand{\zmz}{\mathbb{Z} / m \mathbb{Z} }
\newcommand{\zpz}{\mathbb{Z} / p \mathbb{Z} }
\newcommand{\dbs}{\\~\\}
\newcommand{\zdex}{\mathbb{Z} [x]}
\newcommand{\qdex}{\mathbb{Q} [x]}
\newcommand{\rdex}{\mathbb{R} [x]}
\newcommand{\zdei}{\mathbb{Z} [i]}
\newcommand{\Fp}{\mathbb{F}_p}
\newcommand{\Fpn}{\mathbb{F}_{p^n}}
\newcommand{\Fq}{\mathbb{F}_q}
\newcommand{\Fpdex}{\mathbb{F}_p [x]}
\newcommand{\Fqdex}{\mathbb{F}_q [x]}
\newcommand{\evn}{espace vectoriel normé }
\newcommand{\ssi}{\textit{si et seulement si} }
\newcommand{\edm}{\textit{$(E,d)$} }
\newcommand{\fdm}{\textit{$(F,d_F)$} }


\newcommand{\Mod}[1]{\ (\mathrm{mod}\ #1)}
\newtheorem{mydef}{Définition}
\newtheorem{thm}{Théorème}
\newtheorem{prop}{Proposition}
\newtheorem{cor}{Corollaire}
\newtheorem{lem}{Lemme}
\newtheorem{rap}{Rappel}
\newtheorem{rem}{Remarque}
\newtheorem{propriete}{Propriétés}

%debut du texte
\begin{flushleft}

\textit{Note de l'auteur, c'est un résumé des notes de cours d'analyse 3, pas tous les théorèmes, définitions, et autres y sont écrites, seules celles que je veux retenir et qui ne sont pas évidentes.}\dbs

\textbf{(Cours 2)}

\begin{mydef} 
Soit $(E,d)$ un espace métrique et $A \subset E$.
\begin{enumerate}
\item Un point $a \in E$ est dit adhérent à $A$ si tout voisinage de $a$ rencontre $A$.
\item On note $\bar{A} = $ l'ensemble des points adhérent à $A$.
\item Un point $x \in A$ est dit intérieur à $A$ s'il existe une boule ouverte centré en $x$ et contenue dans $A$.
\item On note $\mathring{A} = $ l'ensemble des points intérieurs de $A$.
\end{enumerate}
\end{mydef}

\begin{mydef}
Soit $(E,d)$ un espace métrique et $A \subset E$. $A$ est dense si tout point de $E$ est adhérent à $A$.
\end{mydef}

\begin{thm} 
Soit $(E,d)$ un espace métrique et $A \subset E$.
\begin{enumerate}
\item $\bar{A}$ est le plus petit fermé contenant $A$.
\item $\mathring{A}$ est le plus grand ouvert contenu dans $A$.
\end{enumerate}
\end{thm}

\textbf{(Cours 3)}

\begin{thm} 
Soit $(E,d_E)$ et $(F,d_F)$ 2 espaces métriques et $f: E \longrightarrow F$ une application. Alors les 3 propriétés suivantes sont équivalentes:
\begin{enumerate}
\item L'application $f$ est continue sur $E$.
\item Pour tout ouvert $U$ de $F$, $ f^{-1} (U) $ est un ouvert de $E$.
\item Pour tout fermé $B$ de $F$, $ f^{-1} (B) $ est un fermé de $E$.
\end{enumerate}
\end{thm}



\textbf{(Cours 4)}
\begin{mydef}
Soit \edm et $(U_i)_{i \in I}$ une famille d'ouvert de $E$. On dit que la famille $(U_i)_{i \in I}$ est un \underline{recouvrement} ouvert de $E$, si $E = \bigcup\limits_{i \in I} U_i$. On appelle un \underline{sous-recouvrement} une sous-famille $(U_i)_{i \in J}$, avec $J \subset I$ fini et $E = \bigcup\limits_{i \in J} U_i$.
\end{mydef}
 
\begin{mydef}
On dit que \edm est un espace métrique \underline{compact} si tout recouvrement ouvert de $E$ contient un sous-recouvrement fini.
\end{mydef} 
 
\begin{rem}
\edm compact $\Leftrightarrow$ $\forall (U_i)_{i \in I}$ une famille d'ouvert tel que $E = \bigcup\limits_{i \in I} U_i$, $\exists J \subset I$ fini vérifiant $E = \bigcup\limits_{i \in J}$\\$\Leftrightarrow$ De tout recouvrement ouvert de $E$, on peut extraire un sous-recouvrement fini.
\end{rem}

\begin{rem}
On peut définir les espaces métriques compacts en utilisant les fermés.
\end{rem}

\begin{thm}
\edm est un compact \ssi toutes familles de fermés non vides, qui est stables par intersection finie, possède une intersection non vide.
\end{thm}

\begin{mydef}
Soit \edm un espace métrique et $K \subset E$ une partie non vide de $E$. On dit que $K$ est compact \ssi $(K, d_K)$ est un espace métrique compact, avec $d_K$ est la restriction de la distance $d$ à $K$.
\end{mydef}


\begin{thm}
Si \edm un espace métrique, $K \subset E$ compact et $F \subset K$ un fermé, alors $F$ est compact.
\end{thm}

%\bigcup\limits_{i \in I}
\begin{propriete}
Soit \edm un espace métrique.
\begin{enumerate}
\item Si $K_1$ et $K_2$ sont 2 compacts de $E$, alors $K_1 \bigcup K_2$ est aussi un compact de $E$.
\item Si $(K_i)_{i \in I}$ est une famille de compacts de $E$ et $\bigcap\limits_{i \in I} K_i \neq \emptyset$, alors $\bigcap\limits_{i \in I} K_i$ est un compact de $E$.
\end{enumerate}
\end{propriete}

\begin{thm}
Soit $(E,d_E)$ et $(F,d_F)$ deux espaces métriques, et $f: E \longrightarrow F$ une application continue. Si $K \subset E$ compact, alors $f(K)$ est un compact de $F$. En particulier, si $E$ est compact et $f$ surjective, alors $F$ est compact.
\end{thm}


\begin{cor}
Si $f: E \longrightarrow F$ est une application continue bijective et $E$ un espace métrique compact, alors $f$ est un homéomorphisme.
\end{cor}

\underline{\textbf{Théorème Bolzano-Weirstrass}}
\begin{thm}
Soit $E$ un espace métrique. Il y a équivalence entre:
\begin{enumerate}
\item L'espace $E$ est compact.
\item Toute suite de point de $E$ possède un point adhérent.
\item Toute suite de point de $E$ possède une sous-suite convergente dans $E$.
\end{enumerate}
\end{thm}

\begin{rap}
Soit $(U_i)_{i \in I}$ un recouvrement ouvert de $E$. Le nombre $\rho > 0$ est dit \underline{nombre de Lebesgue} du recouvrement $(U_i)_{i \in I}$ si $\forall x \in E$, $B(x, \rho )$ est contenue dans certain $U_i$.
\end{rap}

\begin{lem}
Soit $E$ un espace métrique. Si toute suite de points de $E$ possède une valeur d'adhérence, alors tout recouvrement ouvert de $E$ admet un nombre de Lebesgue
\end{lem}

\begin{rap}
Le diamètre: $$\delta (E) := \sup\limits_{x \in E , y \in E} d(x,y)$$
\end{rap}

\begin{thm}
Si $E$ est un espace métrique compact, alors il est de diamètre fini.
\end{thm}

\begin{thm}
Soit \edm un espace métrique compact et $(x_n)_n$ une suite de point de \edm . Si $(x_n)_n$ possède une seule valeur d'adhérence, alors $(x_n)_n$ est convergente
\end{thm}

\begin{mydef}
Soit \edm un espace métrique. On dit que $E$ est précompact si, $\forall \epsilon > 0$, il existe un recouvrement de $E$ par un nombre fini de parties de diamètre inférieur à $\epsilon$.
\end{mydef}

\begin{rem}
On peut montrer l'équivalence suivante:
\begin{enumerate}
\item $E$ est précompacte.
\item $\forall \epsilon > 0$, on peut recouvrir $E$ par un nombre fini de boules ouvertes de rayon $\epsilon$.
\end{enumerate}
\end{rem}

\begin{thm}
Tout espace métrique compact est précompact.
\end{thm}

\begin{rem}
La réciproque est fausse, on verra plus tard qu'un espace métrique est compacte \ssi il est précompact et complet.
\end{rem}

\begin{mydef}
Soit \edm un espace métrique. On dit qu'il est \underline{séparable} \ssi il existe $B \subset E$ une partie de $E$ qui est dénombrable et dense dans $E$. C'est-à-dire $\bar{B} = E$, avec $B \subset E$ dénombrable.
\end{mydef}

\begin{thm}
Tout espace métrique compact est séparable
\end{thm}


\begin{thm}
Si $E$ et $F$ sont des espaces métriques compacts, alors le produit $E \times F$ est compact.
\end{thm}

\begin{thm}
Le produit d'une famille d'espaces métriques compacts est compact.
\end{thm}





%---------------------------------------------------------------------------------------------------------

\textbf{(Cours 5)}

(Suite de Cauchy)
\begin{mydef}
Soit $(E,d)$ un espace métrique. Une suite $(x_n) \subset E$ est dite de Cauchy si:\\
$\forall \epsilon > 0$ , $\exists n_0 \in \mathbb{N}$, $\forall n$, $m \geq n_0$ $d(x_n , x_m) \leq \epsilon$.
\end{mydef}

\begin{rem}
$(x_n)$ est de cauchy $\Leftrightarrow$ $d(x_n, x_m) \longrightarrow 0$ quand $n,m \longrightarrow \infty$\\$\Leftrightarrow$ $\lim\limits_{n \rightarrow \infty} diam (P_n) = 0$ où $P_n =  \{x_k ; k \geq n\}$.
\end{rem}

\begin{prop}
(Premières propriétés des suites de Cauchy). Soit $(E,d)$ un espace métrique.
\begin{enumerate}
\item Toute suite convergente est une suite de Cauchy.
\item Toute suite de Cauchy est bornée.
\item Toute suite extraite $(y_n)$ d'une suite de Cauchy $(x_n)$ est de Cauchy.
\item Si $(x_n)$ est une suite de Cauchy admettant une sous-suite convergente, alors $(x_n)$ est convergente.
\end{enumerate}
\end{prop}

\begin{cor}
Une suite de Cauchy est convergente \ssi elle a une valeur d'adhérence.
\end{cor}


\begin{rem} 
Souvent, pour montrer qu'une suite de Cauchy est convergente, on montre qu'elle possède une valeur d'adhérence.
\end{rem}

\begin{mydef} 
Un espace métrique \edm est dit complet si toute suite de Cauchy de \edm est convergente dans \edm.
\end{mydef}

\begin{rem} 
\begin{enumerate}
\item La notion d'espace complet n'est pas topologique, c'est-à-dire que la notion n'est pas invariante par homéomorphisme. En effet $\mathbb{R}$ et $]-1,1[$ sont homéomorphe, mais $\mathbb{R}$ est complet et  $]-1,1[$ ne l'est pas.
\item Soit \edm un espace métrique et $F \subset E$. $F$ est complet si l'espace métrique $(F, d_F)$ est complet, où $d_F$ est la métrique induite par $d$ sur $F$, c'est-à-dire $d = d|_F$ .Dans la suite, on note $d$ pour $d_F$: la restriction de $d$ à $F$.
\end{enumerate}
\end{rem}

\begin{thm}
Soit \edm un espace métrique \underline{complet} et $F \subset E$. Alors \fdm est complet \ssi $F$ est fermé de $E$.
\end{thm}

\begin{rem} 
$]0,1[ \subset \mathbb{R}$ n'est pas complet, car $]0,1[$ n'est pas fermé. De même, $\mathbb {Q} \subset \mathbb{R}$ n'est pas complet, donc $\mathbb{Q}$ n'est pas fermé dans $\mathbb{R}$
\end{rem}


\begin{prop} 
Soit \edm un espace métrique et $F \subset E$. Alors si $F$ est complet, alors $F$ est fermé.
\end{prop}

\begin{thm}
Un produit fini ou dénombrable d'espaces métriques complets est complet.
\end{thm}

\begin{cor}
$( \mathbb{R}^n, \lVert . \rVert)$ est complet, $\forall n \geq 1$. 
\end{cor}







%\begin{} \end{}
.\\
.\\
.\\
.\\
.\\
.\\
.\\
.\\
.\\
.\\
.\\
.\\
.\\
.\\
.\\
.\\
.\\
.\\
.\\
.\\
.\\
.\\
\textbf{(Cours 6)}


\begin{thm}
Soit $(E,d)$ un espace métrique \underline{complet} et $F \subset E$. Alors $(F,d)$ est complet \ssi $F$ est fermé de $E$.
\end{thm}


\textbf{(Cours 8)}
\begin{cor} Toutes les normes sur un espace vectoriel normé sont équivalentes. Et elles sont topologiquement équivalentes. (Voir théorème 3.1 et corollaire 3.3).
\end{cor}

\begin{thm}
Soit $E$ un espace vectoriel normé de dimension finie et $ \{e_1, ..., e_n \}$ une base de $E$. Alors l'application $$T: \mathbb{R}^n \longrightarrow E$$ $$x = (x_1, ..., x_n) \mapsto x_1 e_1 + ... + x_n e_n$$ est une homéomorphisme.
\end{thm}

\begin{cor} 
Dans un espace vectoriel normé, tout sous-espace de dimension finie est fermé
\end{cor}

\begin{thm}
Si $E$ est un espace vectoriel normé de dimension finie, alors la boule unité fermé $\bar{B} (0,1)$ est compacte.
\end{thm}

\begin{thm}
Si $E$ est un \evn de dimension finie, alors $K \subset E$ est compact de $E$ \ssi $K$ est fermé borné.
\end{thm}

(Théorème de Riez
\begin{thm}
Soit $E$ un \evn . Si la boule unité fermée $\bar{B}(0,1)$ est compacte, alors $E$ est de dimension finie.
\end{thm}

\begin{rq}
En fait, s'il existe une boule fermée et compacte, alors l'espace $E$ est de dimension finie. Suite de la remarque cours 8 page 3 du pdf.
\end{rq}


\underline{Théorème de Hahn-Banach}
\begin{thm}
Soit $E$ un \evn , $F \subset E$ un sous-espace vectoriel de dimension finie et $x \in E$. Alors il existe un point $v \in F$ tel que: $$d(x,F) = \lVert x - v \rVert$$
\end{thm}


\begin{thm}
Soit $E$ un \evn et $F \subset E$ un sous-espace vectoriel. Alors Si $F$ est de dimension finie, alors $F$ est fermé.
\end{thm}

\underline{Notation:}
Si $\bar{x} \in E/F$, alors $\lVert \bar{x} \rVert := inf \lVert y \rVert$, pour $y \in \bar{x}$ (pour y est en dessous du inf..., voir page 5 cours 8)

\begin{thm}
L'application $\bar{x} \in E/F \mapsto \lVert \bar{x} \rVert = inf \lVert y \rVert$ définit une norme sur $E/F$ pour laquelle l'application ... trop long à écrire, voir thm 4.3
\end{thm}

\begin{thm}
Soit $E$ et $F$ 2 \evn et $f: E \longrightarrow F$ une application linéaire de rang fini. ALors $f$ est continue sur $E$ \ssi son noyau est fermé dans E.
\end{thm}

Théorème de Hahn-Banach
\begin{thm}
Soit $E$ un \evn sur $\mathbb{R}$, $H \subset E$ un sous-espace vectoriel et $f: H \longrightarrow \mathbb{R}$ une forme linéaire continue sur $H$. Alors il existe une forme linéaire continue $F$ définie sur $E$ qui prolonge $f$ et de même norme c-a-d $F: E \longrightarrow \mathbb{R}$ continue $$F(x) = f(x), \enspace \forall x \in H$$ $$\lVert F \rVert = \lVert f \rVert $$
\end{thm}


\textbf{(Cours 9)}



\begin{cor}
Soit $E$ un \evn sur $\mathbb{R}$. ALors $E^1 \neq \emptyset$. De plus si $x_0 \in E$ non nul, alors il existe une forme linéaire continue $f_0$ tel que $f_0 (x_0) = \lVert x_0 \rVert$ et $\lVert f_0 \rVert = 1$
\end{cor}

\begin{cor}
Soit $E$ un \evn . Alors l'application ... dur a écrire voir coro 4.7 page 3 cours 9.
\end{cor}

\begin{mydef}
Soit $E$ un \evn . 
\begin{enumerate}
\item Un hyperplan de E est un ensemble de la forme: $$ H = \{ x \in E | f(x) = \alpha \}$$ où $f \in E^*$ non nulle et $\alpha \in \mathbb{R}$ ($f$ n'est pas nécessairement continue). On dit que $H$ est d'équation $f(x) = \alpha$.
\item L'hyperplan $H$ est fermé \ssi $f$ est continue.
\item Une partie $C \subset E$ est dite convexe \ssi $\forall x, y \in C$ et $\forall \lambda , \mu \in \mathbb{R}$ ($\lambda + \mu = 1$), on a $\lambda x + \mu y \in C$ avec (pas sure du reste)
\end{enumerate}
\end{mydef}












































\end{flushleft}

\end{document}

%\\~\\ pour un double enter
%\begin{} \end{}
%... = 	\ldots
% plus grand ou égale = \geq
% somme produit \prod_{i = 1}^{k} (1-\frac{1}{p_i})
% congrus \equiv et modulo \bmod
% barre au dessus \overline{ }.
% isomorphe = \simeq

%$\lVert x \rVert$ = les barre de norme