\documentclass[12pt,a4paper]{article}

\usepackage[left=1.5cm,right=1.5cm,top=1.5cm,bottom=1.5cm]{geometry}
\usepackage[T1]{fontenc}
\usepackage[utf8]{inputenc}
\usepackage[french]{babel}
\usepackage{amssymb}
\newtheorem{theorem}{Theorem}
\newtheorem{definition}{Definition}[section]
\usepackage{setspace}
\onehalfspacing
\usepackage{amsthm}
\usepackage{amsmath}
\begin{document}

%nouvelle commande

\newcommand{\znz}{\mathbb{Z} / n \mathbb{Z} }
\newcommand{\zmz}{\mathbb{Z} / m \mathbb{Z} }
\newcommand{\zpz}{\mathbb{Z} / p \mathbb{Z} }
\newcommand{\dbs}{\\~\\}
\newcommand{\zdex}{\mathbb{Z} [x]}
\newcommand{\qdex}{\mathbb{Q} [x]}
\newcommand{\rdex}{\mathbb{R} [x]}
\newcommand{\zdei}{\mathbb{Z} [i]}
\newcommand{\Fp}{\mathbb{F}_p}
\newcommand{\Fpn}{\mathbb{F}_{p^n}}
\newcommand{\Fq}{\mathbb{F}_q}
\newcommand{\Fpdex}{\mathbb{F}_p [x]}
\newcommand{\Fqdex}{\mathbb{F}_q [x]}
\newcommand{\evn}{espace vectoriel normé }
\newcommand{\ssi}{\textit{si et seulement si} }
\newcommand{\edm}{\textit{$(E,d)$} }
\newcommand{\fdm}{\textit{$(F,d_F)$} }
\newcommand{\nrm}{\lVert . \rVert }
\newcommand{\vertiii}[1]{{\left\vert\kern-0.25ex\left\vert\kern-0.25ex\left\vert #1 
    \right\vert\kern-0.25ex\right\vert\kern-0.25ex\right\vert}}
\newcommand{\fef}{\textit{$f:E \longrightarrow F$} }
\newcommand{\fuf}{\textit{$f:U \longrightarrow F$} }



\newcommand{\Mod}[1]{\ (\mathrm{mod}\ #1)}
\newtheorem{mydef}{Définition}
\newtheorem{thm}{Théorème}
\newtheorem{prop}{Proposition}
\newtheorem{cor}{Corollaire}
\newtheorem{lem}{Lemme}
\newtheorem{rap}{Rappel}
\newtheorem{rem}{Remarque}
\newtheorem{propriete}{Propriétés}

%debut du texte
\begin{flushleft}

\textit{Note de l'auteur, c'est un résumé des notes de cours d'analyse 3, pas tous les théorèmes, définitions, et autres y sont écrites, seules celles que je veux retenir et qui ne sont pas évidentes.}\dbs

\textbf{\underline{Chapitre 1}}\\~\\


\textbf{(Cours 2)}

\begin{mydef} 
Soit $(E,d)$ un espace métrique et $A \subset E$.
\begin{enumerate}
\item Un point $a \in E$ est dit adhérent à $A$ si tout voisinage de $a$ rencontre $A$.
\item On note $\bar{A} = $ l'ensemble des points adhérent à $A$.
\item Un point $x \in A$ est dit intérieur à $A$ s'il existe une boule ouverte centré en $x$ et contenue dans $A$.
\item On note $\mathring{A} = $ l'ensemble des points intérieurs de $A$.
\end{enumerate}
\end{mydef}

\begin{mydef}
Soit $(E,d)$ un espace métrique et $A \subset E$. $A$ est dense si tout point de $E$ est adhérent à $A$.
\end{mydef}

\begin{thm} 
Soit $(E,d)$ un espace métrique et $A \subset E$.
\begin{enumerate}
\item $\bar{A}$ est le plus petit fermé contenant $A$.
\item $\mathring{A}$ est le plus grand ouvert contenu dans $A$.
\end{enumerate}
\end{thm}

\textbf{(Cours 3)}

\begin{thm} 
Soit $(E,d_E)$ et $(F,d_F)$ 2 espaces métriques et $f: E \longrightarrow F$ une application. Alors les 3 propriétés suivantes sont équivalentes:
\begin{enumerate}
\item L'application $f$ est continue sur $E$.
\item Pour tout ouvert $U$ de $F$, $ f^{-1} (U) $ est un ouvert de $E$.
\item Pour tout fermé $B$ de $F$, $ f^{-1} (B) $ est un fermé de $E$.
\end{enumerate}
\end{thm}



\textbf{(Cours 4)}
\begin{mydef}
Soit \edm et $(U_i)_{i \in I}$ une famille d'ouvert de $E$. On dit que la famille $(U_i)_{i \in I}$ est un \underline{recouvrement} ouvert de $E$, si $E = \bigcup\limits_{i \in I} U_i$. On appelle un \underline{sous-recouvrement} une sous-famille $(U_i)_{i \in J}$, avec $J \subset I$ fini et $E = \bigcup\limits_{i \in J} U_i$.
\end{mydef}
 
\begin{mydef}
On dit que \edm est un espace métrique \underline{compact} si tout recouvrement ouvert de $E$ contient un sous-recouvrement fini.
\end{mydef} 
 
\begin{rem}
\edm compact $\Leftrightarrow$ $\forall (U_i)_{i \in I}$ une famille d'ouvert tel que $E = \bigcup\limits_{i \in I} U_i$, $\exists J \subset I$ fini vérifiant $E = \bigcup\limits_{i \in J}$\\$\Leftrightarrow$ De tout recouvrement ouvert de $E$, on peut extraire un sous-recouvrement fini.
\end{rem}

\begin{rem}
On peut définir les espaces métriques compacts en utilisant les fermés.
\end{rem}

\begin{thm}
\edm est un compact \ssi toutes familles de fermés non vides, qui est stables par intersection finie, possède une intersection non vide.
\end{thm}

\begin{mydef}
Soit \edm un espace métrique et $K \subset E$ une partie non vide de $E$. On dit que $K$ est compact \ssi $(K, d_K)$ est un espace métrique compact, avec $d_K$ est la restriction de la distance $d$ à $K$.
\end{mydef}


\begin{thm}
Si \edm un espace métrique, $K \subset E$ compact et $F \subset K$ un fermé, alors $F$ est compact.
\end{thm}

%\bigcup\limits_{i \in I}
\begin{propriete}
Soit \edm un espace métrique.
\begin{enumerate}
\item Si $K_1$ et $K_2$ sont 2 compacts de $E$, alors $K_1 \bigcup K_2$ est aussi un compact de $E$.
\item Si $(K_i)_{i \in I}$ est une famille de compacts de $E$ et $\bigcap\limits_{i \in I} K_i \neq \emptyset$, alors $\bigcap\limits_{i \in I} K_i$ est un compact de $E$.
\end{enumerate}
\end{propriete}

\begin{thm}
Soit $(E,d_E)$ et $(F,d_F)$ deux espaces métriques, et $f: E \longrightarrow F$ une application continue. Si $K \subset E$ compact, alors $f(K)$ est un compact de $F$. En particulier, si $E$ est compact et $f$ surjective, alors $F$ est compact.
\end{thm}


\begin{cor}
Si $f: E \longrightarrow F$ est une application continue bijective et $E$ un espace métrique compact, alors $f$ est un homéomorphisme.
\end{cor}

\underline{\textbf{Théorème Bolzano-Weirstrass}}
\begin{thm}
Soit $E$ un espace métrique. Il y a équivalence entre:
\begin{enumerate}
\item L'espace $E$ est compact.
\item Toute suite de point de $E$ possède un point adhérent.
\item Toute suite de point de $E$ possède une sous-suite convergente dans $E$.
\end{enumerate}
\end{thm}

\begin{rap}
Soit $(U_i)_{i \in I}$ un recouvrement ouvert de $E$. Le nombre $\rho > 0$ est dit \underline{nombre de Lebesgue} du recouvrement $(U_i)_{i \in I}$ si $\forall x \in E$, $B(x, \rho )$ est contenue dans certain $U_i$.
\end{rap}

\begin{lem}
Soit $E$ un espace métrique. Si toute suite de points de $E$ possède une valeur d'adhérence, alors tout recouvrement ouvert de $E$ admet un nombre de Lebesgue
\end{lem}

\begin{rap}
Le diamètre: $$\delta (E) := \sup\limits_{x \in E , y \in E} d(x,y)$$
\end{rap}

\begin{thm}
Si $E$ est un espace métrique compact, alors il est de diamètre fini.
\end{thm}

\begin{thm}
Soit \edm un espace métrique compact et $(x_n)_n$ une suite de point de \edm . Si $(x_n)_n$ possède une seule valeur d'adhérence, alors $(x_n)_n$ est convergente
\end{thm}

\begin{mydef}
Soit \edm un espace métrique. On dit que $E$ est précompact si, $\forall \epsilon > 0$, il existe un recouvrement de $E$ par un nombre fini de parties de diamètre inférieur à $\epsilon$.
\end{mydef}

\begin{rem}
On peut montrer l'équivalence suivante:
\begin{enumerate}
\item $E$ est précompacte.
\item $\forall \epsilon > 0$, on peut recouvrir $E$ par un nombre fini de boules ouvertes de rayon $\epsilon$.
\end{enumerate}
\end{rem}

\begin{thm}
Tout espace métrique compact est précompact.
\end{thm}

\begin{rem}
La réciproque est fausse, on verra plus tard qu'un espace métrique est compacte \ssi il est précompact et complet.
\end{rem}

\begin{mydef}
Soit \edm un espace métrique. On dit qu'il est \underline{séparable} \ssi il existe $B \subset E$ une partie de $E$ qui est dénombrable et dense dans $E$. C'est-à-dire $\bar{B} = E$, avec $B \subset E$ dénombrable.
\end{mydef}

\begin{thm}
Tout espace métrique compact est séparable
\end{thm}


\begin{thm}
Si $E$ et $F$ sont des espaces métriques compacts, alors le produit $E \times F$ est compact.
\end{thm}

\begin{thm}
Le produit d'une famille d'espaces métriques compacts est compact.
\end{thm}





%---------------------------------------------------------------------------------------------------------

\textbf{(Cours 5)}

(Suite de Cauchy)
\begin{mydef}
Soit $(E,d)$ un espace métrique. Une suite $(x_n) \subset E$ est dite de Cauchy si:\\
$\forall \epsilon > 0$ , $\exists n_0 \in \mathbb{N}$, $\forall n$, $m \geq n_0$ $d(x_n , x_m) \leq \epsilon$.
\end{mydef}

\begin{rem}
$(x_n)$ est de cauchy $\Leftrightarrow$ $d(x_n, x_m) \longrightarrow 0$ quand $n,m \longrightarrow \infty$\\$\Leftrightarrow$ $\lim\limits_{n \rightarrow \infty} diam (P_n) = 0$ où $P_n =  \{x_k ; k \geq n\}$.
\end{rem}

\begin{prop}
(Premières propriétés des suites de Cauchy). Soit $(E,d)$ un espace métrique.
\begin{enumerate}
\item Toute suite convergente est une suite de Cauchy.
\item Toute suite de Cauchy est bornée.
\item Toute suite extraite $(y_n)$ d'une suite de Cauchy $(x_n)$ est de Cauchy.
\item Si $(x_n)$ est une suite de Cauchy admettant une sous-suite convergente, alors $(x_n)$ est convergente.
\end{enumerate}
\end{prop}

\begin{cor}
Une suite de Cauchy est convergente \ssi elle a une valeur d'adhérence.
\end{cor}


\begin{rem} 
Souvent, pour montrer qu'une suite de Cauchy est convergente, on montre qu'elle possède une valeur d'adhérence.
\end{rem}

\begin{mydef} 
Un espace métrique \edm est dit \underline{complet} si toute suite de Cauchy de \edm est convergente dans \edm.
\end{mydef}

\begin{rem} 
\begin{enumerate}
\item La notion d'espace complet n'est pas topologique, c'est-à-dire que la notion n'est pas invariante par homéomorphisme. En effet $\mathbb{R}$ et $]-1,1[$ sont homéomorphe, mais $\mathbb{R}$ est complet et  $]-1,1[$ ne l'est pas.
\item Soit \edm un espace métrique et $F \subset E$. $F$ est complet si l'espace métrique $(F, d_F)$ est complet, où $d_F$ est la métrique induite par $d$ sur $F$, c'est-à-dire $d = d|_F$ .Dans la suite, on note $d$ pour $d_F$: la restriction de $d$ à $F$.
\end{enumerate}
\end{rem}

\begin{thm}
Soit \edm un espace métrique \underline{complet} et $F \subset E$. Alors \fdm est complet \ssi $F$ est fermé de $E$.
\end{thm}

\begin{rem} 
$]0,1[ \subset \mathbb{R}$ n'est pas complet, car $]0,1[$ n'est pas fermé. De même, $\mathbb {Q} \subset \mathbb{R}$ n'est pas complet, donc $\mathbb{Q}$ n'est pas fermé dans $\mathbb{R}$
\end{rem}


\begin{prop} 
Soit \edm un espace métrique et $F \subset E$. Alors si $F$ est complet, alors $F$ est fermé.
\end{prop}

\begin{thm}
Un produit fini ou dénombrable d'espaces métriques complets est complet.
\end{thm}

\begin{cor}
$( \mathbb{R}^n, \lVert . \rVert)$ est complet, $\forall n \geq 1$. 
\end{cor}

\begin{thm}
Un espace métrique \edm est compact \ssi il est précompact et complet
\end{thm}

\underline{\textbf{Théorème de Cantor}}
\begin{thm}
Soit \edm un espace métrique. Alors \edm est complet \ssi pour toute suite $(F_n)$ d'ensembles fermés non vides tels que
\begin{enumerate}
\item $F_1 \supset F_2 \supset F_3 \supset F_4 \supset  \ldots F_n \supset F_{n+1} \supset \ldots$  (on dit que $(F_n)$ est décroissante),
\item $\lim\limits_{n \rightarrow \infty} diam(F_n) = 0$ ($diam(F_n) = $ diamètre de $F_n$)
\end{enumerate}
On a l'intersection des $(F_n)$ qui contient un et un seul point.
\end{thm}

\underline{\textbf{Théorème de Baire}}
\begin{thm}
Soit \edm un espace métrique complet et $(\theta_n)_{n \in \mathbb{N}}$ une suite d'ouverts dense dans $E$ ($\bar{\theta_n} = E$, $\forall n \in \mathbb{N}$). Alors $\bigcap\limits_{n \in \mathbb{N}} \theta_n$ est aussi dense dans $E$.
\end{thm}

\begin{cor}
Soit \edm un espace métrique complet et $(F_n)$ une suite de fermés d'intérieur vide ($\mathring{F_n} = \emptyset$, $\forall n \geq 1$). Alors  $\bigcup\limits_{n \geq 1} F_n$ est d'intérieur vide.
\end{cor}

\begin{cor}
Soit \edm un espacae métrique complet et $(F_n)$ une suite de fermés telle que $E = \bigcup\limits_{n} F_n$. Alors il existe $n_0 \in \mathbb(N)$ tel que $\mathring{F_{n_0}} \neq \emptyset$.
\end{cor}

\begin{thm}
Si $f$ est une fonction continue du compact $K$ dans $\mathbb{R}$, alors $f$ est bornée sur $K$ et atteint sur $K$ sa borne supérieure et sa borne inférieure.
 \end{thm}


\underline{\textbf{Théorème de Heine}}
\begin{thm}
Soit $(E,d_E)$ un espace métrique compact, \fdm un espace métrique et $f: E \longrightarrow F$ une application continue. ALors $f$ est \underline{uniformément} continue.
 \end{thm}

\begin{mydef}
Soit $(E,d_E)$ et \fdm deux espaces métrique et $f: E \longrightarrow F$ une application. On dit que $F$ est une\underline{contraction} (ou une application contractante) s'il existe une constante $k$ : $0 < k < 1$ vérifiant:
$$ d_F (f(x), f(y)) \leq k d_E (x,y), \forall x, y \in E$$
 \end{mydef}


\underline{\textbf{Théorème du point fixe de Banach}}
\begin{thm}
Soit \edm un espace métrique \underline{complet} et $f: E \longrightarrow E$ une \underline{contraction}. Alors il existe un unique point fixe de $f$, c'est-à-dire il existe un unique $a \in E$ tel que $f(a) = a$.
 \end{thm}



\textbf{(Cours 6)}


\begin{rem}
Soit $(K,d)$ un espace métrique compact et \edm un espace métrique complet. On note $C(K,E)$ l'ensemble des fonctions continues de $K$ dans $E$. On définit l'application $d$ sur $C(K,E) $: $$ d(f,g) = \sup\limits_{x \in K} d(f(x), g(x))$$ Cette borne  supérieure est atteinte en au moins un point de $K$.
\end{rem}

\begin{thm}
L'application $d$ définit une distance sur $C(K,E)$ appelée distance de la convergence uniforme.
 \end{thm}


\begin{thm}
Si $K$ est compact et $E$ complet, alors $C(K,E)$ muni de la distance de convergence uniforme est \underline{complet}.
 \end{thm}


\begin{rem}
Si $(f_n)_{n \in \mathbb{N}}$ est une suite de fonctions continues ($f_n \in C(K,E)$), et si $(f_n)_n$ converge vers $f$ pour la distance de la convergence uniforme, alors la limite de $f$ est continue. Inversement, si $(f_n)_n$ converge simplement et si $f$ est continue, la convergence n'est pas nécessairement uniforme.
 \end{rem}


\begin{thm}
Soit $(f_n)$ une suite croissante de fonctions continues de l'espace compact $K$ dans $\mathbb{R}$, $f: K \longrightarrow \mathbb{R}$ une fonction continue. On suppose que $(f_n)_n$ converge simplement vers $f$. Alors $(f_n)_n$ converge uniformément vers $f$.
 \end{thm}


\begin{thm}
Si $H \subset C(K,E)$ compact et $x \in K$, alors l'ensemble $$H(x) = \{ f(x), f \in H \}$$ est compact dans $E$.
 \end{thm}


\begin{mydef}
Une partie $H$ de $C(K,E)$ est dite \underline{équicontinue} au point $x$ de $K$ si, pour tout $ \epsilon > 0$, il existe un voisinage $V$ de $x$ dans $K$ tel que, $$d(f(x),f(y)) < \epsilon$$ $\forall y \in V$ et $\forall f \in H$. 
 \end{mydef}


\begin{thm}
Si $H \subset C(K,E)$ une partie compacte, alors $H$ est équicontinue en tout point de $K$.
 \end{thm}


\begin{thm}
Si $H \subset C(K,E)$ une partie relativement compact, alors l'ensemble $H(x) = \{ f(x), f \in H \}$ est relativement compact dans $E$ et $H$ est équicontinue en chaque point de $K$.
\end{thm}

\underline{\textbf{Théorème de Ascoli-Arzela}}
\begin{thm}
Soit $K,d)$ un espace métrique compact, \edm un espace métrique complet et $H \subset C(K,E)$ une partie non vide. On suppose que:
\begin{enumerate}
\item $H(x)$ est relativement compact pour tout $x \in K$;
\item $H$ est équicontinue en chaque point $x \in K$.
\end{enumerate}
Alors $H$ est relativement compact dans $C(K,E)$
\end{thm}

\textbf{\underline{Chapitre 2 - Analyse linéaire dans les espaces de Banach}}\\~\\

\textbf{(Cours 7)}

\begin{mydef}
Soit $E$ un espace vectoriel sur $\mathbb{R}$. On appelle une norme sur $E$ une application $\lVert . \rVert$ de $E$ dans $[0, \infty[$ vérifiant:
\begin{enumerate}
\item $ \lVert x \rVert = 0  \Leftrightarrow x = 0$.
\item $\lVert \lambda x \rVert = | \lambda | \lVert x \rVert$ , pour tout $\lambda \in \mathbb{R}$ et tout $x \in E$.
\item $\lVert \lambda x+y \rVert \leq \lVert x \rVert + \lVert y \rVert$, $\forall x \in E, \forall y \in E$.
\end{enumerate}
Un espace vectoriel muni d'une norme est appelé un espace \underline{vectoriel normé}. La norme est habituellement noté  $\lVert . \rVert$.
\end{mydef}

\begin{prop}
Si $\lVert . \rVert$ est une norme sur $E$, alors l'application $(x,y) \in E \mapsto \lVert x-y \rVert$ est une distance sur $E$.
\end{prop}

(remarque incomplète - voir page 2-3 du cours 7 pour les détails)
\begin{rem}
D'après la proposition précédente, tout espace normé est métrique. Une suite $(x_n)_n$ de $E$ est convergente, si $(x_n)_n$ converge pour la distance associée à $\lVert . \rVert$. De même, une application $f: E \longrightarrow F$, est continu en $x  \in E$, où $E$ et $F$ sont deux espaces normés.
\end{rem}

\begin{thm}
Si $(E, \nrm )$ est un espace vectoriel normé, alors les applications
\begin{enumerate}
\item $(x,y) \in E \times E \longmapsto x+y \in E$, est continue.
\item $(\lambda, x) \in \mathbb{R} \times E \longmapsto \lambda x \in E$, est continue.
\item $x \in E \longmapsto \lVert x \rVert \in [0, \infty [ $, est continue.
\end{enumerate}
\end{thm}


\begin{mydef}
On appelle un espace de \underline{Banach} tout espace vectoriel normé complet pour la distance associée à la norme. 
\end{mydef}

\begin{rap}
Soit $E$ et $F$ deux espaces vectoriels sur $\mathbb{R}$ et $f: E \longrightarrow F$ une application. Alors $f$ est linéaire \ssi $$f( \lambda x + \mu y) = \lambda f(x) + \mu f(y)$$
$\forall x,y \in E$ et $\lambda , \mu \in \mathbb{R}$.
\end{rap}

\begin{thm}
Soit $f:E \longrightarrow F$ une application linéaire de l'espace vectoriel normé $(E, \nrm_E)$ dans l'espace vectoriel normé $(F, \nrm_F)$. Les propriétés suivantes sont équivalentes:
\begin{enumerate}
\item L'application linéaire $f$ est continue sur $E$;
\item $f$ est continue en 0;
\item $f$ est uniformément continue sur $E$;
\item $f$ est lipschitzienne;
\item Il existe une constante $k > 0$, telle que $\lVert f(x) \rVert_F \leq k \lVert x \rVert_E$, $\forall x \in E$.
\end{enumerate}
\end{thm}

On note $L(E,F)  =$ l'ensemble des applications linéaires continues de l'espace normé $(E, \nrm_E)$ dans l'espace vectoriel normé $(F, \nrm_F)$.

\begin{rem}
$L(E,F)$ est un espace vectoriel sur $\mathbb(R)$ (évident).
\end{rem}

\begin{mydef}
Soit $f \in L(E,F)$. On appelle la norme de $f$ la constante définie par : $$\vertiii{f} = \sup\limits_{x \in E} \lVert f(x) \rVert_F$$
\end{mydef}

\begin{thm}
Soit $f \in L(E,F)$. Alors on a: 
\begin{align*}
\vertiii{f} : &= \sup\limits_{x \in E , \lVert x \rVert_E \leq 1} \lVert f(x) \rVert_F\\
	       &=\sup\limits_{x \in E , \lVert x \rVert_E = 1} \lVert f(x) \rVert_F\\
		&= \sup\limits_{x \in E , \lVert x \rVert_E \neq 0} frac{\lVert f(x) \rVert_F}{\lVert x \rVert_E}\\
		&= \min \{ k > 0; \lVert f(x) \rVert_F \leq k \lVert x \rVert_E , \forall x \in E \}
\end{align*}
\end{thm}

\begin{thm}
L'application $f \in L(E,F) \mapsto \vertiii{f}$ est une norme sur $L(E,F)$. De plus, si $F$ est un espace de Banach, alors $(L(E,F), \vertiii{.})$ est aussi un espace de Banach.
\end{thm}

\begin{thm}
Soit $E$, $F$ et $G$ trois espaces vectoriels normés et $f:E \longrightarrow F$ et $g: F \longrightarrow G$ deux applications linéaires continues. Alors l'application $g \circ f : E \longrightarrow G$ est linéaire et continue et $$ \vertiii{g \circ f} \leq \vertiii{g} . \vertiii{f}$$ En particulier, si $E = F = G$, alors $ L(E):= L(E,E)$ est une algèbre de Banach.
\end{thm}

(Banach)
\begin{thm}
Soit $E$ et $F$ deux espaces de Banach et $f:E \longrightarrow F$ une application linéaire continue et bjective de $E$ sur $F$. Alors $f^{-1}$ est continue de $F$ sur $E$.
\end{thm}

\begin{mydef}
Une application \underline{linéaire} \fef est dite isomorphisme si elle est continue, bijective et son inverse $f^{-1}$ est aussi continue.
\end{mydef} 



\textbf{(Cours 8)}
\begin{cor} Toutes les normes sur un espace vectoriel normé (pas nécessairement de dimension fini) sont équivalentes. Et elles sont topologiquement équivalentes. (Voir théorème 3.1 et corollaire 3.3).
\end{cor}

\begin{thm}
Soit $E$ un espace vectoriel normé de dimension finie et $ \{e_1, ..., e_n \}$ une base de $E$. Alors l'application $$T: \mathbb{R}^n \longrightarrow E$$ $$x = (x_1, ..., x_n) \mapsto x_1 e_1 + ... + x_n e_n$$ est une homéomorphisme.
\end{thm}

\begin{cor} 
Dans un espace vectoriel normé, tout sous-espace de dimension finie est fermé
\end{cor}

\begin{thm}
Si $E$ est un espace vectoriel normé de dimension finie, alors la boule unité fermé $\bar{B} (0,1)$ est compacte.
\end{thm}

\begin{thm}
Si $E$ est un \evn de dimension finie, alors $K \subset E$ est compact de $E$ \ssi $K$ est fermé borné.
\end{thm}

\textbf{\underline{(Théorème de Riez)}}
\begin{thm}
Soit $E$ un \evn . Si la boule unité fermée $\bar{B}(0,1)$ est compacte, alors $E$ est de dimension finie.
\end{thm}

\begin{rem}
En fait, s'il existe une boule fermée et compacte, alors l'espace $E$ est de dimension finie. Suite de la remarque cours 8 page 3 du pdf.
\end{rem}


\begin{thm}
Soit $E$ un \evn , $F \subset E$ un sous-espace vectoriel de dimension finie et $x \in E$. Alors il existe un point $v \in F$ tel que: $$d(x,F) = \lVert x - v \rVert$$
\end{thm}


\begin{thm}
Soit $E$ un \evn et $F \subset E$ un sous-espace vectoriel. Alors Si $F$ est de dimension finie, alors $F$ est fermé.
\end{thm}

\underline{Notation:}
Si $\bar{x} \in E/F$, alors $\lVert \bar{x} \rVert := inf \lVert y \rVert$, pour $y \in \bar{x}$ (pour y est en dessous du inf..., voir page 5 cours 8)

\begin{thm}
L'application $\bar{x} \in E/F \mapsto \lVert \bar{x} \rVert = inf \lVert y \rVert$ définit une norme sur $E/F$ pour laquelle l'application
\begin{align*}
\Pi: E &\longrightarrow E/F\\
        x	&\mapsto \bar{x} 
\end{align*}
est continue, de plus, $\forall x  \in E$, $$\lVert \bar{x} \rVert = \lVert \Pi (x) \rVert = d(x,F)$$
La norme définie sur $E/F$ est dite la \underline{norme quotient}.
\end{thm}

\begin{thm}
Soit $E$ et $F$ 2 \evn et $f: E \longrightarrow F$ une application linéaire de rang fini. Alors $f$ est continue sur $E$ \ssi son noyau est fermé dans E.
\end{thm}

\underline{Forme analytique du théorème de Hahn - Banach}\\
Soit $E$ un espace vectoriel normé sur $\mathbb{R}$. On rappelle qu'une forme linéaire est une application linéaire définie sur $E$ et à valeurs dans $\mathbb{R}$. Le théorème de Hahn-Banach concerne le prolongement d'une forme linéaire définie sur un sous-espace vectoriel de $E$ en une forme linéaire définie sur $E$ tout entier.\\

\textbf{\underline{Théorème de Hahn-Banach}}
\begin{thm}
Soit $E$ un \evn sur $\mathbb{R}$, $H \subset E$ un sous-espace vectoriel et $f: H \longrightarrow \mathbb{R}$ une forme linéaire continue sur $H$. Alors il existe une forme linéaire continue $F$ définie sur $E$ qui prolonge $f$ et de même norme c-a-d $F: E \longrightarrow \mathbb{R}$ continue $$F(x) = f(x), \enspace \forall x \in H$$ $$\lVert F \rVert = \lVert f \rVert $$
\end{thm}


\textbf{\underline{Lemme de Zorn}}
\begin{lem}
Tout ensemble ordonné, inductif, non vide, admet un élément maximal.
\end{lem}

\textbf{(Cours 9)}



\begin{cor}
Soit $E$ un \evn sur $\mathbb{R}$. ALors $E^1 \neq \emptyset$. De plus si $x_0 \in E$ non nul, alors il existe une forme linéaire continue $f_0$ tel que $f_0 (x_0) = \lVert x_0 \rVert$ et $\lVert f_0 \rVert = 1$
\end{cor}

\begin{cor}
Soit $E$ un \evn . Alors l'application ... dur a écrire voir coro 4.7 page 3 cours 9.
\end{cor}

\begin{mydef}
Soit $E$ un \evn . 
\begin{enumerate}
\item Un \underline{hyperplan} de E est un ensemble de la forme: $$ H = \{ x \in E | f(x) = \alpha \}$$ où $f \in E^*$ non nulle et $\alpha \in \mathbb{R}$ ($f$ n'est pas nécessairement continue). On dit que $H$ est d'équation $f(x) = \alpha$.
\item L'hyperplan $H$ est \underline{fermé} \ssi $f$ est continue.
\item Une partie $C \subset E$ est dite \underline{convexe} \ssi $\forall x, y \in C$ et $\forall \lambda , \mu \in \mathbb{R}$ ($\lambda + \mu = 1$), on a $\lambda x + \mu y \in C$ avec $\lambda + \mu = 1$ où $\lambda \geq 1$ et $\mu \geq 0$.


\item Soit $A$ et $B$ deux parties de $E$. On dit que l'hyperplan $H = \{ f = \alpha \}$ sépare $A$ et $B$ au sens large si on a $ A  \subset \{ x \in E | f(x) \leq \alpha \}$ et $B \subset \{ x \in E | f(x) \geq \alpha \}$

\item On dit que $H = \{ f = \alpha \}$ sépare $A$ et $B$ au sens strict si on a qu'il $\exists \epsilon >0$ tel que $ A \subset \{ x \in E | f(x) \leq \alpha - \epsilon \}$ et $B \subset \{ x \in E | f(x) \geq \alpha + \epsilon \}$
\end{enumerate}
\end{mydef}


\underline{Hahn-Banach, première forme géométrique}



\begin{thm}
Soit $E$ un espace vectoriel normal et $A$ et $B$ deux parties convexes, non vides et disjointes. Si $A$ est un ouvert, alors il existe un hyperplan fermé qui sépare $A$ et $B$ au sens large.
\end{thm}

\underline{Hahn-Banach, deuxième forme géométrique}
\begin{thm}
Soit $E$ un espace vectoriel normé et $A$ et $B$ deux parties convexes non vides et disjointes. Si $A$ est fermé et $B$ compact, alors il existe un hyperplan fermé qui sépare $A$ et $B$ au sens \underline{stricte}
\end{thm}

\begin{cor}
Soit $\qquad F \nsubseteq E$ un sous-espace vectoriel fermé de $E$ ($F \neq E$). Alors il existe une forme $f \in E^\prime$ non nulle telle que $f_{|_F} = 0$ (c'est-à-dire $f(x) = 0, \forall x \in F$)
\end{cor}

\textbf{\underline{Chapitre 3 - Calcul différentiel dans les espaces de Banach}}\\~\\

Soit $E$ et $F$ deux espaces vectoriels normés, $U \subset E$ un ouvert non vide et $\fuf$ une application.

\begin{mydef}
Soit $x_0 \in U$. On dit que $f$ est différentiable en $x_0$ s'il existe une application linéaire continue $l \in L(E,F)$ tel que  $$ \lim\limits_{h \rightarrow 0_E} \frac{ \lVert f(x_0 + h ) - f(x_0) - l(h) \rVert_F}{ \lVert h  \rVert_E} = 0$$
\end{mydef}

\begin{rem}
Autrement dit, $f$ est différentiable en $x_0$ , s'il existe une application linéaire continue $l \in L(E,F)$, $R > 0$ et $\varepsilon : B_E (0_E, R) \longrightarrow \mathbb{R}$, où la fonction $\varepsilon$ vérifie: $\lim\limits_{h \rightarrow 0_E} \varepsilon (h) = 0$, telle que $$\lVert f(x_0 + h ) - f(x_0) - l(h) \rVert_F =  \varepsilon (h)  \lVert h  \rVert_E$$
Il suffit en effet de poser $\varepsilon (0_E) = 0$ et si $R > 0$ est choisi de sorte que la boule $B(x_0 , R)$ de centre $x_0$ et de rayon $R > 0 $ soit incluse dans $U$ et $$ \varepsilon (h) :=  \frac{ \lVert f(x_0 + h ) - f(x_0) - l(h) \rVert_F}{ \lVert h  \rVert_E}$$
si $h \in B(0_E , R)$
\end{rem}

\begin{prop}
Si $f$ est différentiable en $x_0$, alors $f$ est nécessairement continue en $x_0$.
\end{prop}

\underline{Notation:}\\
L'application $l$ dans la définition précédente (avant la remarque et la proposition) dépend du point $x_0$. Il faut donc adopter une notation faisant apparaître le point $x_0$. On utilise la notation $l = df(x_0)$. Donc $df(x_0)$ est elle-même une application. On l'appelle la différentielle de $f$ en $x_0$.

\begin{mydef}
On dit que $f$ est \underline{différentielle} sur $U$ si elle est différentiable en tout point $x \in U$. On appelle la différentielle de $f$ la fonction
\begin{align*}
df : U & \longrightarrow L(E,F)\\
      x	& \longmapsto df(x)
\end{align*}
Si de plus cette application est ocntinue, on dit que $f$ est de classe $C^1$ (ou continument différentiable).
\end{mydef}

\begin{rem}
Il ne faut pas confondre $df$ et sa valeur en $x$, $df(x)$. La continuité de $df$ n'a rien à voir avec la continuité de $df(x)$ (cette dernière est toujours continue, mais $df$ ne l'est pas en général). Aussi $df(x)$ est linéaire de $E$ dans $F$, mais $df$ n'est pas linéaire en général de $U$ dans $L(E,F)$.
\end{rem}

(à voir exemples pages 7-8 cours 9)\\

\begin{thm}
Si $\fuf$ et $g: U \longrightarrow F$ sont différentiables, alors $f+g $ est différentiable et $d(f+g) = df + dg$, c'est-à-dire $d(f+g)(x)(h) = df(x)(h) + dg(x)(h)$.\\L'application $( \lambda f )$ est différentiable $\forall \lambda \in \mathbb(R)$ et $d( \lambda f) = \lambda df$.
\end{thm}

\begin{rem}
L'ensemble des applications différentiables sur $U$ est un espace vectoriel et la différentiation $d$ est une application linéaire de cet espace vectoriel dans l'ensemble des applications de $U$ dans $L(E,F)$.
\end{rem}

\textbf{(Cours 10)}\\~\\

\begin{thm}
Soit $E$, $F$ et $G$ trois espaces vectoriels normés, $U \subset E$ un ouvert et $ V \subset F$ un ouvert. Si $\fuf$ est différentiable en un point $x \in U$ et si $g: V \subset F \longrightarrow G$ est différentiable en $f(x) \in V$, alors la fonction composée $g \circ f : U \longrightarrow G$ est différentiable en $x$ et $$ d( g \circ f ) (x) = dg(f(x)) . (df(x).h)$$ $\forall x \in U$ et $\forall h \in E$.
\end{thm}

\begin{rem}
C'est un résultat très important, c'est la "dérivation en chaine" ou "règle de dérivation des fonctions composées. Voir exemple page 1 cours 10.
\end{rem}

\underline{2. Fonctions définies sur un espace produit}\\~\\
Si 
\begin{align*}
f: U \subset E_1 \times E_2 \times \cdots \times E_n & \longrightarrow F\\
					x & \longmapsto f(x)
\end{align*}
est différentiable. Alors les applications partielles $$y_i \longmapsto f(x_1, \cdots , x_{i-1} , y_i, x_{i+1} , \cdots , x_n)$$ sont différentiables et si on note $ d_i f(x_1, \cdots , x_n)$ leur différentielles en $x_i$, alors $$df(x_1, \cdots , x_n) = \sum\limits_{i=1}^n d_i f(x_1, \cdots , x_n) . h_i$$ avec $h \in E_1 \times \cdots \times E_n$ et $h = (h_1, \cdots , h_n)$.

\underline{Différentielles en dimension finie}\\~\\
On suppose que $E = \mathbb{R}^n$ et $F = \mathbb{R}^k$. Soit $B = \{ e_1, \cdots , e_n \}$, la base canonique de $\mathbb{R}^n$ et $\varepsilon = \{ \varepsilon_1, \cdots , \varepsilon_n \}$ la base canonique de $\mathbb{R}^k$. Alors si $f: U \subset \mathbb {R}^n \longrightarrow \mathbb{R}^k$ est différentiable, alors les applications partielles:
\begin{align*}
g_i: \Theta_i(x) & \longrightarrow F\\
					t & \longmapsto  f(x_1, \cdots , x_{i-1} , t , x_{i+1} , \cdots , x_n)
\end{align*}
$$g_i(t) = f(x+(t-x_i) e_i$$ où $$ \Theta_i (x) = \{ t \in \mathbb{R} | (x_1, \cdots , x_{i-1} , t , x_{i+1} , \cdots , x_n) \in U \}.$$ On note $\dfrac{df}{dx_i} (x)$ la dérivée de $g_i$ en $x_i$. Alors on a: (...) texte à lire page 4.

\begin{thm}
$f: u \subset \mathbb{R}^n \longrightarrow \mathbb{R}^k$ est de classe $C^1$ (continument différentiable) \ssi ses dérivées partielles existent et sont continues sur $U$.
\end{thm} 

(à lire, texte du théorème des accroissements fini page 4)\\~\\

\begin{thm}
Soit $I \subset \mathbb{R}$ un intervalle ouvert, $F$ un espace vectoriel normé sur $\mathbb{R}$ et $f: I \longrightarrow \mathbb{R}$ une fonction dérivable. On suppose qu'il existe $k > 0$ tel que $\lVert f'(t) \rVert_F \leq k$, $\forall t \in I$. Alors $$ \lVert f(x) - f(y) \rVert_F \leq k |x-y|$$ $\forall x \in I$ et $\forall y \in I$.
\end{thm}

\begin{thm}
Soit $f: I \subset \mathbb{R} \longrightarrow F$ une fonction dérivable sur l'intervalle ouvert $I$ et à valeures dans l'espace de Banach $F$ sur $\mathbb{R}$. On suppose qu'il existe une fonction $\varphi : I \longrightarrow \mathbb{R}$ dérivable telle que $$\lVert \varphi^\prime (t) \rVert_F \leq \varphi^\prime (t), \forall t\in I$$
$$ \text{Alors } \lVert f(x) - f(y) \rVert_F \leq | \varphi(x) - \varphi(y) | \text{, } \forall x \in I \text{, } \forall y \in I$$
\end{thm}

\begin{thm}
Soit $f: U \subset E \longrightarrow F$ une fonction différentiable sur l'ouvert \underline{convexe} U. On suppose qu'il existe une constate $k > 0$ tel que $\vertiii{df(x)} \leq k$, $\forall x \in U$ $$ \text{Alors } \lVert f(x) - f(y) \rVert_F \leq k \lVert x-y \rVert_E \text{, } \forall x \in U \text{, } \forall y \in U$$
\end{thm}

\begin{rem}
En fait on peut montrer l'inégalité plus fine suivante: $$ \text{Alors } \lVert f(x) - f(y) \rVert \leq \sup\limits_{t \in [0,1]} \vertiii{df(y + (1-t)x} . \lVert x-y \rVert_E$$
\end{rem}

%\begin{} \end{}




%Rendu ici


























\end{flushleft}

\end{document}

%\\~\\ pour un double enter
%\begin{} \end{}
%... = 	\ldots
% plus grand ou égale = \geq
% somme produit \prod_{i = 1}^{k} (1-\dfrac{1}{p_i})
% congrus \equiv et modulo \bmod
% barre au dessus \overline{ }.
% isomorphe = \simeq

%$\lVert x \rVert$ = les barre de norme