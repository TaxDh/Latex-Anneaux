\documentclass[12pt,a4paper]{article}

\usepackage[left=1.5cm,right=1.5cm,top=1.5cm,bottom=1.5cm]{geometry}
\usepackage[T1]{fontenc}
\usepackage[utf8]{inputenc}
\usepackage[french]{babel}
\usepackage{amssymb}
\newtheorem{theorem}{Theorem}
\newtheorem{definition}{Definition}[section]
\usepackage{setspace}
\onehalfspacing
\usepackage{amsthm}
\usepackage{amsmath}
\begin{document}

%nouvelle commande

\newcommand{\znz}{\mathbb{Z} / n \mathbb{Z} }
\newcommand{\zmz}{\mathbb{Z} / m \mathbb{Z} }
\newcommand{\zpz}{\mathbb{Z} / p \mathbb{Z} }
\newcommand{\dbs}{\\~\\}
\newcommand{\zdex}{\mathbb{Z} [x]}
\newcommand{\qdex}{\mathbb{Q} [x]}
\newcommand{\rdex}{\mathbb{R} [x]}
\newcommand{\zdei}{\mathbb{Z} [i]}
\newcommand{\Fp}{\mathbb{F}_p}
\newcommand{\Fpn}{\mathbb{F}_{p^n}}
\newcommand{\Fq}{\mathbb{F}_q}
\newcommand{\Fpdex}{\mathbb{F}_p [x]}
\newcommand{\Fqdex}{\mathbb{F}_q [x]}
\newcommand{\evn}{espace vectoriel normé }
\newcommand{\ssi}{\textit{si et seulement si} }
\newcommand{\edm}{\textit{$(E,d)$} }
\newcommand{\fdm}{\textit{$(F,d_F)$} }
\newcommand{\nrm}{\lVert . \rVert }
\newcommand{\vertiii}[1]{{\left\vert\kern-0.25ex\left\vert\kern-0.25ex\left\vert #1 
    \right\vert\kern-0.25ex\right\vert\kern-0.25ex\right\vert}}
\newcommand{\fef}{\textit{$f:E \longrightarrow F$} }
\newcommand{\fuf}{\textit{$f:U \longrightarrow F$} }




\newcommand{\Mod}[1]{\ (\mathrm{mod}\ #1)}
\newtheorem{mydef}{Définition}
\newtheorem{thm}{Théorème}
\newtheorem{prop}{Proposition}
\newtheorem{cor}{Corollaire}
\newtheorem{lem}{Lemme}
\newtheorem{rap}{Rappel}
\newtheorem{rem}{Remarque}
\newtheorem{propriete}{Propriétés}

%debut du texte
\begin{flushleft}

\textit{Note de l'auteur, c'est un résumé des notes de cours d'analyse 3, pas tous les théorèmes, définitions, et autres y sont écrites, seules celles que je veux retenir et qui ne sont pas évidentes.}\dbs

\textbf{(Cours 2)}

\begin{mydef} 
Soit $(E,d)$ un espace métrique et $A \subset E$.
\begin{enumerate}
\item Un point $a \in E$ est dit adhérent à $A$ si tout voisinage de $a$ rencontre $A$.
\item On note $\bar{A} = $ l'ensemble des points adhérent à $A$.
\item Un point $x \in A$ est dit intérieur à $A$ s'il existe une boule ouverte centré en $x$ et contenue dans $A$.
\item On note $\mathring{A} = $ l'ensemble des points intérieurs de $A$.
\end{enumerate}
\end{mydef}

\begin{mydef}
Soit $(E,d)$ un espace métrique et $A \subset E$. $A$ est dense si tout point de $E$ est adhérent à $A$.
\end{mydef}

\begin{thm} 
Soit $(E,d)$ un espace métrique et $A \subset E$.
\begin{enumerate}
\item $\bar{A}$ est le plus petit fermé contenant $A$.
\item $\mathring{A}$ est le plus grand ouvert contenu dans $A$.
\end{enumerate}
\end{thm}

\textbf{(Cours 3)}

\begin{thm} 
Soit $(E,d_E)$ et $(F,d_F)$ 2 espaces métriques et $f: E \longrightarrow F$ une application. Alors les 3 propriétés suivantes sont équivalentes:
\begin{enumerate}
\item L'application $f$ est continue sur $E$.
\item Pour tout ouvert $U$ de $F$, $ f^{-1} (U) $ est un ouvert de $E$.
\item Pour tout fermé $B$ de $F$, $ f^{-1} (B) $ est un fermé de $E$.
\end{enumerate}
\end{thm}

\textbf{(Cours 4)}

\underline{\textbf{Théorème Bolzano-Weirstrass}}
\begin{thm}
Soit $E$ un espace métrique. Il y a équivalence entre:
\begin{enumerate}
\item L'espace $E$ est compact.
\item Toute suite de point de $E$ possède un point adhérent.
\item Toute suite de point de $E$ possède une sous-suite convergente dans $E$.
\end{enumerate}
\end{thm}

\begin{rem}
Vu sur un site internet:\\~\\
dans R et plus generalement dans un espace vectoriel norme de dimension finie: 
A compact <=>A fermé et borné 

pour completer: \\~\\

1.dans tout metrique : A compact => A ferme et borné 
avec reciproque fausse en général \\~\\

2.dans un espace vectoriel normé E: 
sphere unite fermée compacte => E de dimension finie ( th de Riesz) 
\end{rem}
\textbf{(Cours 5)}

\begin{prop}
(Premières propriétés des suites de Cauchy). Soit $(E,d)$ un espace métrique.
\begin{enumerate}
\item Toute suite convergente est une suite de Cauchy.
\item Toute suite de Cauchy est bornée.
\item Toute suite extraite $(y_n)$ d'une suite de Cauchy $(x_n)$ est de Cauchy.
\item Si $(x_n)$ est une suite de Cauchy admettant une sous-suite convergente, alors $(x_n)$ est convergente.
\end{enumerate}
\end{prop}

\begin{cor}
Une suite de Cauchy est convergente \ssi elle a une valeur d'adhérence.
\end{cor}

\begin{thm}
Soit $f:E \longrightarrow F$ une application linéaire de l'espace vectoriel normé $(E, \nrm_E)$ dans l'espace vectoriel normé $(F, \nrm_F)$. Les propriétés suivantes sont équivalentes:
\begin{enumerate}
\item L'application linéaire $f$ est continue sur $E$;
\item $f$ est continue en 0;
\item $f$ est uniformément continue sur $E$;
\item $f$ est lipschitzienne;
\item Il existe une constante $k > 0$, telle que $\lVert f(x) \rVert_F \leq k \lVert x \rVert_E$, $\forall x \in E$.
\end{enumerate}
\end{thm}


\end{flushleft}

\end{document}

%\\~\\ pour un double enter
%\begin{} \end{}
%... = 	\ldots
% plus grand ou égale = \geq
% somme produit \prod_{i = 1}^{k} (1-\frac{1}{p_i})
% congrus \equiv et modulo \bmod
% barre au dessus \overline{ }.
% isomorphe = \simeq

%$\lVert x \rVert$ = les barre de norme