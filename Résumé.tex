\documentclass[]{article}

\usepackage[T1]{fontenc}
\usepackage[utf8]{inputenc}
\usepackage[french]{babel}
\usepackage{amssymb}
\newtheorem{theorem}{Theorem}
\newtheorem{definition}{Definition}[section]
\usepackage{setspace}
\onehalfspacing
\usepackage{amsthm}
\begin{document}

%nouvelle commande

\newcommand{\znz}{$\mathbb{Z} / n \mathbb{Z}$ }
\newcommand{\dbs}{\\~\\}
\newtheorem{mydef}{Définition}
\newtheorem{thm}{Théorème}
\newtheorem{prop}{Proposition}

%debut du texte
\begin{flushleft}
\Large
\textbf{\underline{Chapitre 1}}\\~\\

\begin{prop} Soit $A$ un anneau. Alors l'ensemble $A^{n \times n}$ des matrices de tailles $n \times n$ sur $A$ est un anneau.\end{prop} 
\begin{prop} \znz est un anneau.
\end{prop}
\begin{mydef} Un sous-anneau $B$ de l'anneau $A$ est un sous-groupe additif de $A$ tel que:
\begin{enumerate}
\item $ \forall a,b \in B, ab \in B $
\item $ 1 \in B$
\end{enumerate} 
\end{mydef}

\end{flushleft}


 
\end{document}

%\\~\\ pour un double enter
%\begin{} \end{}