\documentclass[]{article}

\usepackage[T1]{fontenc}
\usepackage[utf8]{inputenc}
\usepackage[french]{babel}
\usepackage{amssymb}
\newtheorem{theorem}{Theorem}
\newtheorem{definition}{Definition}[section]
\usepackage{setspace}
\onehalfspacing
\usepackage{amsthm}
\begin{document}

%nouvelle commande

\newcommand{\znz}{$\mathbb{Z} / n \mathbb{Z}$ }
\newcommand{\dbs}{\\~\\}
\newtheorem{mydef}{Définition}
\newtheorem{thm}{Théorème}
\newtheorem{prop}{Proposition}
\newtheorem{cor}{Corollaire}
\newtheorem{lem}{Lemme}

%debut du texte
\begin{flushleft}
\Large

\textit{Note de l'auteur, c'est un résumé des notes de cours de théorie des anneaux, pas tous les théorèmes, définitions, et autres y sont écrites, seules celles que je veux retenir et qui ne sont pas évidentes.}\dbs

\textbf{\underline{Chapitre 1}}\\~\\

\begin{prop} Soit $A$ un anneau. Alors l'ensemble $A^{n \times n}$ des matrices de tailles $n \times n$ sur $A$ est un anneau.\end{prop} 

\begin{prop} \znz est un anneau.
\end{prop}

\begin{mydef} Un sous-anneau $B$ de l'anneau $A$ est un sous-groupe additif de $A$ tel que:
\begin{enumerate}
\item $ \forall a,b \in B, ab \in B $
\item $ 1 \in B$
\end{enumerate} 
\end{mydef}

\begin{cor} 
L'intersection de tous les sous-anneaux de l'anneau A est l'ensemble $ \{ n \cdot 1 | n \in \mathbb{Z} \} $, avec la notation usuelles des groupes additifs.
\end{cor}

\begin{mydef}
Si $A$ est un anneau, on dit qu'un élément $a$ de $A$ est inversible s'il existe $b$ de $A$ tel que $ab=ba=1$. Un tel élément $b$ est alors unique et est appelé inverse de $a$.
\end{mydef}

\begin{mydef}
L'ensembles des éléments inversibles d'un anneau $A$ est noté $U(A)$. 
\end{mydef}

\begin{prop}
$U(A)$ est un groupe sous la multiplication.
\end{prop}

\end{flushleft}


 
\end{document}

%\\~\\ pour un double enter
%\begin{} \end{}