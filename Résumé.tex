\documentclass[12pt,a4paper]{article}

\usepackage[left=1.5cm,right=1.5cm,top=1.5cm,bottom=1.5cm]{geometry}
\usepackage[T1]{fontenc}
\usepackage[utf8]{inputenc}
\usepackage[french]{babel}
\usepackage{amssymb}
\newtheorem{theorem}{Theorem}
\newtheorem{definition}{Definition}[section]
\usepackage{setspace}
\onehalfspacing
\usepackage{amsthm}
\begin{document}

%nouvelle commande

\newcommand{\znz}{\mathbb{Z} / n \mathbb{Z} }
\newcommand{\zmz}{\mathbb{Z} / m \mathbb{Z} }
\newcommand{\zpz}{\mathbb{Z} / p \mathbb{Z} }
\newcommand{\dbs}{\\~\\}
\newcommand{\zdex}{\mathbb{Z} [x]}
\newcommand{\zdei}{\mathbb{Z} [i]}
\newcommand{\Mod}[1]{\ (\mathrm{mod}\ #1)}
\newtheorem{mydef}{Définition}
\newtheorem{thm}{Théorème}
\newtheorem{prop}{Proposition}
\newtheorem{cor}{Corollaire}
\newtheorem{lem}{Lemme}
\newtheorem{rap}{Rappel}
\newtheorem{rem}{Remarque}


%debut du texte
\begin{flushleft}

\textit{Note de l'auteur, c'est un résumé des notes de cours de théorie des anneaux, pas tous les théorèmes, définitions, et autres y sont écrites, seules celles que je veux retenir et qui ne sont pas évidentes.}\dbs

\textbf{\underline{Chapitre 1}}\\~\\

\begin{mydef}
Un anneau est dit commutatif si la loi de multiplication est commutative.
\end{mydef}

\begin{prop} Soit $A$ un anneau. Alors l'ensemble $A^{n \times n}$ des matrices de tailles $n \times n$ sur $A$ est un anneau.\end{prop} 

\begin{prop} $\znz$ est un anneau.
\end{prop}

\begin{mydef} Un sous-anneau $B$ de l'anneau $A$ est un sous-groupe additif de $A$ tel que:
\begin{enumerate}
\item $ \forall a,b \in B, ab \in B $
\item $ 1 \in B$
\end{enumerate} 
\end{mydef}

\begin{cor} 
L'intersection de tous les sous-anneaux de l'anneau A est l'ensemble $ \{ n \cdot 1 | n \in \mathbb{Z} \} $, avec la notation usuelles des groupes additifs.
\end{cor}

\begin{mydef}
Si $A$ est un anneau, on dit qu'un élément $a$ de $A$ est inversible s'il existe $b$ de $A$ tel que $ab=ba=1$. Un tel élément $b$ est alors unique et est appelé inverse de $a$.
\end{mydef}

\begin{mydef}
L'ensembles des éléments inversibles d'un anneau $A$ est noté $U(A)$. 
\end{mydef}

\begin{prop}
$U(A)$ est un groupe sous la multiplication.
\end{prop}

\begin{mydef}
Dans un anneau $A$ un diviseur de $0$ est un élément de $a$ de $A$, tel que $a \neq 0$ et :
\begin{enumerate}
\item $ab = 0$ ($a$ est un diviseur de $0$ à gauche).
\item $ba = 0$ ($a$ est un diviseur de $0$ à droite).
\end{enumerate}
\end{mydef}

\begin{mydef}
Un anneau est dit intègre s'il n'a aucun diviseur de $0$.
\end{mydef}

\begin{prop}
Si $a,b,c \in A$, un anneau intègre. Alors $ab = 0 \Rightarrow a=0$ ou $b=0$. De plus, si $a \neq 0$ et $ab = ac \Rightarrow b=c$ et $ba = ca \Rightarrow b = c.$
\end{prop}

\begin{prop}
Soient $A$,$B$ deux anneaux. L'ensemble $A \times B$ muni de l'addition $$ (a,b) + (a',b') = (a+a',b+b')$$ et de la multiplication $$(a,b) \cdot (a',b') = (aa',bb')$$ est un anneau, avec $0_{A \times B} = (0,0)$ et l'élément neutre $1_{A \times B} = (1,1)$. Il est commutatif si et seulement si $A$ et $B$ le sont aussi. L'anneau $A \times B$ n'est pas intègre.
\end{prop}

\begin{mydef}
On appelle $A \times B$ l'aneau produit de $A$ et $B$.
\end{mydef}

\begin{prop} 
$U(A \times B) = U(A) \times U(B) $
\end{prop}

\begin{mydef}
Un homomorphisme d'anneau $f:A\longrightarrow B$ est une fonction tel que:
\begin{enumerate}
\item $f$ est un homomorphisme de groupes additifs.
\item $\forall a,b \in A$, $f(aa')=f(a)f(a')$.
\item $f(1_A) = 1_B$.
\end{enumerate}
\end{mydef}

\begin{mydef}
Un idéal dans un anneau $A$ est un sous-ensemble $I$ tel que:
\begin{enumerate}
\item $I$ est un sous-groupe additif.
\item $\forall a \in A$, $\forall x \in I$, $ax, xa \in I$.
\end{enumerate}
\end{mydef}

\begin{prop}
Le noyau d'un homomorphisme est un idéal.
\end{prop}

\begin{prop}
Soit $I$ un idéal d'un anneau $A$, tel que $I \neq A$. On construit le groupe additif $A/I$, quotient des groupes additifs $A$ et $I$. Alors $A/I$ est un anneau, tel que l'homomorphisme canonique de groupe $A \longrightarrow A/I$ est aussi un homomorphisme d'anneaux.
\end{prop}

\begin{mydef}
On appelle $A/I$ l'anneau quotient de $A$ par l'idéal $I$.
\end{mydef}

\begin{thm}
Il est à la fin de la page 7(chapitre 1), il n'est pas copiable à cause d'une figure. À lire.
\end{thm}

\begin{cor}
Si $f: A\longrightarrow B$ est un isomorphisme d'anneau, on a toujours l'isomorphisme d'anneau $A/ker(f) \simeq f(a)$
\end{cor}

\begin{prop}
L'image et l'image réciproque d'un sous-anneau est un sous-anneau. L'image réciproque d'un idéal est un idéal. Si l'homomorphisme est surjectif, alors l'image d'un idéal est un idéal.
\end{prop}

\begin{prop} 
$\zmz$ est intègre si et seulement si $m$ est premier.
\end{prop}

\begin{prop}
Les éléments inversibles de $\zmz$ sont les $n$ avec $ n \perp m$. 
\end{prop}

\begin{cor}
$U(\zmz )$ est en bijection avec $\{ n | 0 \leqslant n \leqslant m-1, n \perp m  \} $. En particulier $| U(\zmz)| = \varphi (m)$.
\end{cor}

\begin{rap}
$\varphi (m) = | \{ n | 0 \leqslant n \leqslant m-1 \}$. $\varphi$ est appelé l'indicateur d'Euler, ou fonction d'Euler.
\end{rap}

\begin{prop}
Si $m$, $p$ sont premier entre eux, alors $\varphi (m p) = \varphi (m) \varphi (p)$.
\end{prop}

\begin{cor}
Si $m = p_1^{m_1} \ldots p_k^{m_k}$, $p_i$ premiers distincts, alors 
$$ \varphi (m) = \prod_{i = 1}^{k} (p_i^{m_i} - p_i^{m_{i-1}})$$
$$= m \prod_{i = 1}^{k} (1-\frac{1}{p_i}) $$
\end{cor}

\begin{mydef}
La caractéristique d'un anneau $A$ est l'ordre pour la loi additive de l'élément neutre de la loi multiplicative. Par exemple, la caractéristique de $\znz$ est $n$, car $n \cdot 1 = 0$ mod $n$
\end{mydef}

\begin{prop}
Si la caractéristique d'un anneau intègre n'est pas nulle, c'est un nombre premier.
\end{prop}

\begin{prop}
Si $A$ est un anneau commutatif de caractéristique $p$, un nombre premier, alors l'application $F: A \rightarrow A$, $x \mapsto x^p$, est un homomorphisme d'anneaux. On l'appelle l'homomorphisme de Frobenius.
\end{prop}

\begin{mydef}
Un corps est un anneau où tout élément non nul est inversible.
\end{mydef}

\begin{prop}
Un corps est intègre.
\end{prop}

\begin{mydef}
Un sous-corps d'un corps est un sous-anneau qui contient l'inverse de chaque éléments non nuls qu'il contient.
\end{mydef}

\begin{rem}
Un sous-corps est un corps.
\end{rem}

\begin{prop}
L'anneau commutatif $A$ est un corps si et seulement si ses seuls idéaux sont $\{ 0 \}$ et $A$.
\end{prop}

\begin{prop}
$p$ est premiers si et seulement si $\mathbb{Z}/ p \mathbb{Z}$ est un corps.
\end{prop}

\begin{mydef}
Un idéal $I \neq A$ d'un anneau commutatif $A$ est dit maximal si:
$\forall J$ idéal de $A$, $I \subseteq J \subseteq A$, on a $J = I$ ou $J = A$.
\end{mydef}

\begin{prop}
Soit $A$ un anneau commutatif et $I$ un idéal. Alors $I$ est maximal si et seulement si $A/I$ est un corps.
\end{prop}

\begin{cor}
Les idéaux (ou sous-groupes) maximaux de $\mathbb{Z}$ sont les $p \mathbb{Z}$, $p$ premier.
\end{cor}

\begin{prop}
Si un corps $K$ est de caractéristique non nulle, celle-ci étant un nombre premier, et $K$ contient un sous-corps isomorphe à $\zpz$, à savoir son sous-corps premier.
\end{prop}

\begin{prop}
Soit $K$ un corps et $A$ un anneau). Si $f: K \rightarrow A$ est un homomorphisme d'anneaux, alors $f$ est injectif.
\end{prop}

\begin{prop}
L'ensemble $\{ a + b \cdot i| a, b \in \mathbb{Z} \} $ est un sous-anneau de $\mathbb{C}$ dont les élément inversible sont $\{-1, 1, -i, i \}$
\end{prop}

\begin{prop}
Soit $E$ et $A$ un anneau. L'ensemble $A^E$ des fonctions de $E$ dans $A$ est un anneau. La somme et le produit sont défini par: 
\begin{enumerate}
\item $(f+g)(e)= f(e) + g(e)$.
\item $(f \cdot g) (e) = f(e) \cdot g(e)$.
\end{enumerate}
Les éléments neutres pour l'addition et la multiplication sont les fonctions constantes égale, l'une à $0$, l'autre à $1$. 
\end{prop}


\textbf{\underline{Chapitre 2}}\\~\\

\begin{mydef}
Dans un anneau, on dit que $a$ divise $ b $, noté $ a | b $, s'il existe $ c $ tel que $ ac = b $. Alors $ b $ s'appelle multiple de $ a $ et $ a $ un diviseur de  $b $. Note, aucun rapport avec diviseur de $ 0 $.
\end{mydef}

\begin{rem}
\begin{enumerate}
\item si $ a $ est inversible, $ a $ divise n'importe quel $ b $, car $ b = a (a^{-1} b) $
\item si $ a $ divise $ b $ et si $ u $ est inversible, alors $ au $ divise aussi $ b $, car $b = ac \Rightarrow b = (au) (u^{-1} c)$.
\end{enumerate}
\end{rem}

\begin{mydef}
Deux éléments $a$ et $b$ de $A$, anneau commutatif, sont dits associés, s'il existe $u \in A$, $u$ inversible, tel que $ b = au$.
\end{mydef}

\begin{prop}
Si $A$ est intègre, alors $a$ et $b$ associés est une relation d'équivalence. On a\\
\begin{center}
$Aa = Ab \Leftrightarrow a $ et $b$ associés.
\end{center}
\end{prop}

\begin{mydef}
Un élément non nul et non inversible d'un anneau commutatif intègre $A$ est dit irréductible si et seulement si  $\forall b , c \in A$, $a = bc \Rightarrow b $ ou $c$ inversible.
\end{mydef}

\begin{mydef}
Deux éléments $a$ et $b$ d'un anneau commutatif intègre $A$ sont dits premiers entre eux si: $\forall x \in A$, $x $ divise $a$ et $x$ divise $b \Rightarrow x$ inversible.
\end{mydef}

\begin{mydef}
Soient $A$ un anneau intègre et $\sigma : A^* \longrightarrow \mathbb{N}$ une application. L'anneau $A$ est euclidien pour $\sigma$ si :
\begin{enumerate}
\item pour tous $a, b \in A^*$ tel que $a$ divise $b$, on a $\sigma(a) \leq \sigma(b)$.
\item pour tous $a \in A$ et $b \in A^*$, il existe $q$ et $r$ $\in A$ tel que $a = bq + r$ avec $r = 0$ ou $\sigma(r) < \sigma(b)$.
\end{enumerate}
L'application $\sigma$ s'appelle parfois un stahme (ou une valuation).
\end{mydef}

\begin{mydef}
Un anneau commutatif est dit principal si tout idéal est principal.
\begin{enumerate}
\item Un idéal d'un anneau commutatif $A$ est dit principal s'il est de la forme $Aa$, $a \in A$.
\end{enumerate}
\end{mydef}

\begin{rem}
$Aa = $ ensemble des multiples de $A$.
\end{rem}

\begin{thm}
Tout anneau euclidien est principal.
\end{thm}

\begin{mydef}
Soit $A$ un anneau commutatif intègre. Il est dit factoriel si: 
\begin{enumerate}
\item Tout élément $a$ de $A$, qui n'est ni nul ni inversible, est un produit d'éléments irréductibles.
$$ a = p_1 \ldots p_i$$
\item Si pour un élément de $a$ $p_1 \ldots p_n = q_1 \ldots q_m$, alors $m = n$ et il existe une 	permutation $\sigma$ de l'ensemble $\{ 1, \ldots , n \}$ ainsi que des éléments inversibles $u_1 , \ldots , u_n$ tel que $p_i = u_i q_{\sigma (i)}$ pour tout $i$.
\end{enumerate}
\end{mydef}

\begin{mydef}
Un anneau commutatif $A$ satisfait la condition de chaine ascendante si pour toute suite croissante d'idéaux
$$I_1 \subseteq I_2 \subseteq I_3 \subseteq \ldots \subseteq I_n \subseteq \ldots$$
Il existe $r$ tel que $I_s = I_r$, $\forall s \geq r$.
\end{mydef}

\begin{lem}
Un anneau principal satisfait à la condition de chaine ascendante.
\end{lem}

\begin{lem}
Soit $A$ un anneau principal intègre. Alors $a, b$ sont premiers entre eux $\Leftrightarrow \exists x, y \in A$ tel que $ax + by = 1$.
\end{lem}

\begin{lem}
Soit $A$ un anneau commutatif intègre principal. Si $a \bot b$ et si $a$ divise $bc$, alors $a$ divise $c$.
\end{lem}

\begin{thm}
Soit $A$ un anneau commutatif intègre. Si $A$ est principal, $A$ est factoriel.
\end{thm}

\begin{cor}
$\mathbb{Z}$ et $K [x]$ sont factoriels ($K$ corps commutatif).
\end{cor}

\begin{thm}
Si $A$ est un anneau factoriel, alors $A[x]$ est un anneau factoriel.
\end{thm}

\begin{cor}
Si $A$ est un anneau factoriel, alors $A[x_1, \ldots , x_n]$ est un anneau factoriel.
\end{cor}

\begin{lem}
$A[x_1 , \ldots , x_n] \simeq B[x_n]$ où $B = A[x_1 , \ldots , x_{n-1}]$
\end{lem}

\begin{rem}
Le corollaire implique que $\mathbb{Z} [x]$ est factoriel. Les éléments irréductibles de $\mathbb{Z} [x]$ sont les +$p$ ou -$p$, $p$ premier dans $\mathbb{N}$, et les polynômes $P(x) \in \mathbb{Z} [x]$, de degré $\geq 1$, qui sont irréductibles dans $\mathbb{Q} [x]$, et qui sont primitifs. 
\end{rem}

\begin{rem}
On dit que $0 \neq P(x) = a_n x^n + \ldots + a_0 \in \mathbb{Z} [x]$ est primitif si $pgcd(a_n , \ldots , a_0) = 1$
\end{rem}

\begin{mydef}
$A = \zdei = \{ a + bi | a,b \in \mathbb{Z}$\\
$N(a + bi) = a^2 + b^2$.\\
On sait que $z \in A$ est inversible si et seulement si $N(z) = 1 \Leftrightarrow z = 1, -1, i, -i$.
\end{mydef}

\begin{thm}
$A$ est euclidien.
\end{thm}

\begin{cor}
$\zdei$ est principal et factoriel.
\end{cor}

\begin{thm}
Soit $p$ un nombre premier, $p \neq 2$. Alors les conditions suivantes sont équivalentes:
\begin{enumerate}
\item $p \equiv 1 \bmod 4$
\item $-1$ est un carré dans $\zpz$.
\item il existe $a, b \in \mathbb{Z}$ tel que $p = a^2 + b^2$.
\item $p$ n'est pas irréductible dans $\zdei$.
\end{enumerate}
\end{thm}

\begin{cor}
Les éléments irréductibles de $\zdei$ sont:
\begin{enumerate}
\item $1+i$
\item les $p$ premiers dans $\mathbb{N}$ avec $p \equiv 3 \bmod 4$.
\item les $a + bi$, $a - bi$ tel que $a^2 + b^2 $ est premier dans $\mathbb{N}$ et leurs associés.
\end{enumerate}
De plus la décomposition $p = a^2 + b^2$ d'un nombre premier est unique.
\end{cor}

\begin{cor}
Soit $n = \prod p^{n_p}$, pour $p$ premiers distincts (voir chap 2 bis page 5). Alors $n$ est somme de deux carré si et seulement si $\forall p \equiv 3 \bmod 4$, on a $n_p$ pair.
\end{cor}

\underline{Note} : À partir de maintenant on prend $A = \mathbb{Z}$ pour $A[x]$. C'est juste pour se simplifier la vie. Donc on parle ici de $\zdex$.

\begin{mydef}
Un polynôme $p \in \zdex$ est dit primitif si le pgcd de ses coefficient est $1$. Il est en particulier non nul.
\end{mydef}
%Rendu chapitre 2ter



\end{flushleft}


 
\end{document}

%\\~\\ pour un double enter
%\begin{} \end{}
%... = 	\ldots
% plus grand ou égale = \geq
% somme produit \prod_{i = 1}^{k} (1-\frac{1}{p_i})
% congrus \equiv et modulo \bmod