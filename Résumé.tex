\documentclass[12pt,a4paper]{article}

\usepackage[left=1.5cm,right=1.5cm,top=1.5cm,bottom=1.5cm]{geometry}
\usepackage[T1]{fontenc}
\usepackage[utf8]{inputenc}
\usepackage[french]{babel}
\usepackage{amssymb}
\newtheorem{theorem}{Theorem}
\newtheorem{definition}{Definition}[section]
\usepackage{setspace}
\onehalfspacing
\usepackage{amsthm}
\begin{document}

%nouvelle commande

\newcommand{\znz}{\mathbb{Z} / n \mathbb{Z} }
\newcommand{\zmz}{\mathbb{Z} / m \mathbb{Z} }
\newcommand{\zpz}{\mathbb{Z} / p \mathbb{Z} }
\newcommand{\dbs}{\\~\\}
\newtheorem{mydef}{Définition}
\newtheorem{thm}{Théorème}
\newtheorem{prop}{Proposition}
\newtheorem{cor}{Corollaire}
\newtheorem{lem}{Lemme}
\newtheorem{rap}{Rappel}
\newtheorem{rem}{Remarque}


%debut du texte
\begin{flushleft}

\textit{Note de l'auteur, c'est un résumé des notes de cours de théorie des anneaux, pas tous les théorèmes, définitions, et autres y sont écrites, seules celles que je veux retenir et qui ne sont pas évidentes.}\dbs

\textbf{\underline{Chapitre 1}}\\~\\

\begin{prop} Soit $A$ un anneau. Alors l'ensemble $A^{n \times n}$ des matrices de tailles $n \times n$ sur $A$ est un anneau.\end{prop} 

\begin{prop} $\znz$ est un anneau.
\end{prop}

\begin{mydef} Un sous-anneau $B$ de l'anneau $A$ est un sous-groupe additif de $A$ tel que:
\begin{enumerate}
\item $ \forall a,b \in B, ab \in B $
\item $ 1 \in B$
\end{enumerate} 
\end{mydef}

\begin{cor} 
L'intersection de tous les sous-anneaux de l'anneau A est l'ensemble $ \{ n \cdot 1 | n \in \mathbb{Z} \} $, avec la notation usuelles des groupes additifs.
\end{cor}

\begin{mydef}
Si $A$ est un anneau, on dit qu'un élément $a$ de $A$ est inversible s'il existe $b$ de $A$ tel que $ab=ba=1$. Un tel élément $b$ est alors unique et est appelé inverse de $a$.
\end{mydef}

\begin{mydef}
L'ensembles des éléments inversibles d'un anneau $A$ est noté $U(A)$. 
\end{mydef}

\begin{prop}
$U(A)$ est un groupe sous la multiplication.
\end{prop}

\begin{mydef}
Dans un anneau $A$ un diviseur de $0$ est un élément de $a$ de $A$, tel que $a \neq 0$ et :
\begin{enumerate}
\item $ab = 0$ ($a$ est un diviseur de $0$ à gauche).
\item $ba = 0$ ($a$ est un diviseur de $0$ à droite).
\end{enumerate}
\end{mydef}

\begin{mydef}
Un anneau est dit intègre s'il n'a aucun diviseur de $0$.
\end{mydef}

\begin{prop}
Si $a,b,c \in A$, un anneau intègre. Alors $ab = 0 \Rightarrow a=0$ ou $b=0$. De plus, si $a \neq 0$ et $ab = ac \Rightarrow b=c$ et $ba = ca \Rightarrow b = c.$
\end{prop}

\begin{prop}
Soient $A$,$B$ deux anneaux. L'ensemble $A \times B$ muni de l'addition $$ (a,b) + (a',b') = (a+a',b+b')$$ et de la multiplication $$(a,b) \cdot (a',b') = (aa',bb')$$ est un anneau, avec $0_{A \times B} = (0,0)$ et l'élément neutre $1_{A \times B} = (1,1)$. Il est commutatif si et seulement si $A$ et $B$ le sont aussi. L'anneau $A \times B$ n'est pas intègre.
\end{prop}

\begin{mydef}
On appelle $A \times B$ l'aneau produit de $A$ et $B$.
\end{mydef}

\begin{prop} 
$U(A \times B) = U(A) \times U(B) $
\end{prop}

\begin{mydef}
Un homomorphisme d'anneau $f:A\longrightarrow B$ est une fonction tel que:
\begin{enumerate}
\item $f$ est un homomorphisme de groupes additifs.
\item $\forall a,b \in A$, $f(aa')=f(a)f(a')$.
\item $f(1_A) = 1_B$.
\end{enumerate}
\end{mydef}

\begin{mydef}
Un idéal dans un anneau $A$ est un sous-ensemble $I$ tel que:
\begin{enumerate}
\item $I$ est un sous-groupe additif.
\item $\forall a \in A$, $\forall x \in I$, $ax, xa \in I$.
\end{enumerate}
\end{mydef}

\begin{prop}
Le noyau d'un homomorphisme est un idéal.
\end{prop}

\begin{prop}
Soit $I$ un idéal d'un anneau $A$, tel que $I \neq A$. On construit le groupe additif $A/I$, quotient des groupes additifs $A$ et $I$. Alors $A/I$ est un anneau, tel que l'homomorphisme canonique de groupe $A \longrightarrow A/I$ est aussi un homomorphisme d'anneaux.
\end{prop}

\begin{mydef}
On appelle $A/I$ l'anneau quotient de $A$ par l'idéal $I$.
\end{mydef}

\begin{thm}
Il est à la fin de la page 7(chapitre 1), il n'est pas copiable à cause d'une figure. À lire.
\end{thm}

\begin{cor}
Si $f: A\longrightarrow B$ est un isomorphisme d'anneau, on a toujours l'isomorphisme d'anneau $A/ker(f) \simeq f(a)$
\end{cor}

\begin{prop}
L'image et l'image réciproque d'un sous-anneau est un sous-anneau. L'image réciproque d'un idéal est un idéal. Si l'homomorphisme est surjectif, alors l'image d'un idéal est un idéal.
\end{prop}

\begin{prop} 
$\zmz$ est intègre si et seulement si $m$ est premier.
\end{prop}

\begin{prop}
Les éléments inversibles de $\zmz$ sont les $n$ avec $ n \perp m$. 
\end{prop}

\begin{cor}
$U(\zmz )$ est en bijection avec $\{ n | 0 \leqslant n \leqslant m-1, n \perp m  \} $. En particulier $| U(\zmz)| = \varphi (m)$.
\end{cor}

\begin{rap}
$\varphi (m) = | \{ n | 0 \leqslant n \leqslant m-1 \}$. $\varphi$ est appelé l'indicateur d'Euler, ou fonction d'Euler.
\end{rap}

\begin{prop}
Si $m$, $p$ sont premier entre eux, alors $\varphi (m p) = \varphi (m) \varphi (p)$.
\end{prop}

\begin{cor}
Si $m = p_1^{m_1} \ldots p_k^{m_k}$, $p_i$ premiers distincts, alors 
$$ \varphi (m) = \prod_{i = 1}^{k} (p_i^{m_i} - p_i^{m_{i-1}})$$
$$= m \prod_{i = 1}^{k} (1-\frac{1}{p_i}) $$
\end{cor}

\begin{mydef}
La caractéristique d'un anneau $A$ est l'ordre pour la loi additive de l'élément neutre de la loi multiplicative. Par exemple, la caractéristique de $\znz$ est $n$, car $n \cdot 1 = 0$ mod $n$
\end{mydef}

\begin{prop}
Si la caractéristique d'un anneau intègre n'est pas nulle, c'est un nombre premier.
\end{prop}

\begin{prop}
Si $A$ est un anneau commutatif de caractéristique $p$, un nombre premier, alors l'application $F: A \rightarrow A$, $x \mapsto x^p$, est un homomorphisme d'anneaux. On l'appelle l'homomorphisme de Frobenius.
\end{prop}

\begin{mydef}
Un corps est un anneau où tout élément non nul est inversible.
\end{mydef}

\begin{prop}
Un corps est intègre.
\end{prop}

\begin{mydef}
Un sous-corps d'un corps est un sous-anneau qui contient l'inverse de chaque éléments non nuls qu'il contient.
\end{mydef}

\begin{rem}
Un sous-corps est un corps.
\end{rem}

\begin{prop}
L'anneau commutatif $A$ est un corps si et seulement si ses seuls idéaux sont $\{ 0 \}$ et $A$.
\end{prop}

\begin{prop}
$p$ est premiers si et seulement si $\mathbb{Z}/ p \mathbb{Z}$ est un corps.
\end{prop}

\begin{mydef}
Un idéal $I \neq A$ d'un anneau commutatif $A$ est dit maximal si:
$\forall J$ idéal de $A$, $I \subseteq J \subseteq A$, on a $J = I$ ou $J = A$.
\end{mydef}

\begin{prop}
Soit $A$ un anneau commutatif et $I$ un idéal. Alors $I$ est maximal si et seulement si $A/I$ est un corps.
\end{prop}

\begin{cor}
Les idéaux (ou sous-groupes) maximaux de $\mathbb{Z}$ sont les $p \mathbb{Z}$, $p$ premier.
\end{cor}

\begin{prop}
Si un corps $K$ est de caractéristique non nulle, celle-ci étant un nombre premier, et $K$ contient un sous-corps isomorphe à $\zpz$, à savoir son sous-corps premier.
\end{prop}

\begin{prop}
Soit $K$ un corps et $A$ un anneau). Si $f: K \rightarrow A$ est un homomorphisme d'anneaux, alors $f$ est injectif.
\end{prop}

\begin{prop}
L'ensemble $\{ a + b \cdot i| a, b \in \mathbb{Z} \} $ est un sous-anneau de $\mathbb{C}$ dont les élément inversible sont $\{-1, 1, -i, i \}$
\end{prop}

\begin{prop}
Soit $E$ et $A$ un anneau. L'ensemble $A^E$ des fonctions de $E$ dans $A$ est un anneau. La somme et le produit sont défini par: 
\begin{enumerate}
\item $(f+g)(e)= f(e) + g(e)$.
\item $(f \cdot g) (e) = f(e) \cdot g(e)$.
\end{enumerate}
Les éléments neutres pour l'addition et la multiplication sont les fonctions constantes égale, l'une à $0$, l'autre à $1$. 
\end{prop}

%Rendu chapitre 2



\end{flushleft}


 
\end{document}

%\\~\\ pour un double enter
%\begin{} \end{}